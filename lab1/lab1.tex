\documentclass[11pt]{article}
\usepackage{amsmath,amssymb,amsthm}
\usepackage{algorithm}
\usepackage[noend]{algpseudocode} 

%---enable russian----

\usepackage[utf8]{inputenc}
\usepackage[russian]{babel}

% PROBABILITY SYMBOLS
\newcommand*\PROB\Pr 
\DeclareMathOperator*{\EXPECT}{\mathbb{E}}


% Sets, Rngs, ets 
\newcommand{\N}{{{\mathbb N}}}
\newcommand{\Z}{{{\mathbb Z}}}
\newcommand{\R}{{{\mathbb R}}}
\newcommand{\Zp}{\ints_p} % Integers modulo p
\newcommand{\Zq}{\ints_q} % Integers modulo q
\newcommand{\Zn}{\ints_N} % Integers modulo N

% Landau 
\newcommand{\bigO}{\mathcal{O}}
\newcommand*{\OLandau}{\bigO}
\newcommand*{\WLandau}{\Omega}
\newcommand*{\xOLandau}{\widetilde{\OLandau}}
\newcommand*{\xWLandau}{\widetilde{\WLandau}}
\newcommand*{\TLandau}{\Theta}
\newcommand*{\xTLandau}{\widetilde{\TLandau}}
\newcommand{\smallo}{o} %technically, an omicron
\newcommand{\softO}{\widetilde{\bigO}}
\newcommand{\wLandau}{\omega}
\newcommand{\negl}{\mathrm{negl}} 

% Misc
\newcommand{\eps}{\varepsilon}
\newcommand{\inprod}[1]{\left\langle #1 \right\rangle}

 
\newcommand{\handout}[5]{
  \noindent
  \begin{center}
  \framebox{
    \vbox{
      \hbox to 5.78in { {\bf Научно-исследовательская практика} \hfill #2 }
      \vspace{4mm}
      \hbox to 5.78in { {\Large \hfill #5  \hfill} }
      \vspace{2mm}
      \hbox to 5.78in { {\em #3 \hfill #4} }
    }
  }
  \end{center}
  \vspace*{4mm}
}

\newcommand{\lecture}[4]{\handout{#1}{#2}{#3}{Scribe: #4}{Number-Theoretic Functions #1}}

\newtheorem{theorem}{Теорема}
\newtheorem{lemma}{Лемма}
\newtheorem{definition}{Определение}
\newtheorem{corollary}{Следствие}
\newtheorem{fact}{Факт}

% 1-inch margins
\topmargin 0pt
\advance \topmargin by -\headheight
\advance \topmargin by -\headsep
\textheight 8.9in
\oddsidemargin 0pt
\evensidemargin \oddsidemargin
\marginparwidth 0.5in
\textwidth 6.5in

\parindent 0.4in
\parskip 0ex

\begin{document}

\lecture{}{Лето 2020}{}{Толпекин Максим}
 
Посколку $\mu$ как известно является мультипликативной функцией, обращение к 
Теореме 6-4 является законной; этот результат гарантирует, что  $F$ также мультипликативна.
Таким образом, если каноническая факторизация $n$ является $n=p^{k_1}_1 p^{k_2\dots}_2 p^{k_r}_r$,
точка $F(n)$ произведение значений, присвоенных $F$ для простых степеней в этом представлении:
\[
F(n)=F(p^{k_1}_1)F(p^{k_2}_2)\dots F(p^{k_r}_r)=0.
\]
\begin{flushleft}Мы запишем этот результат как \\[5mm] \end{flushleft}
{\setlength{\leftskip}{9mm}
	\setlength{\rightskip}{9mm}
\begin{theorem} 
	\label{Теорема 6--6.}
{\it Для каждого положительного числа $n\geqslant 1$,}
\[\sum_{d|n}\mu(d)=\begin{cases}
1&\text{if $n=1$} \\
0&\text{if $n>1$}
\end{cases}
\]\\
{\it где $d$ является положительными делителями $n$.}\\
\end{theorem}
}

{Для иллюстрации этой теоремы, рассмотрим  $n=10$. Делителями числа  10 являются  1, 2, 5, 10, а искомая сумма равна }\\

\begin{align*}
\sum_{d|10}\mu(d)=\mu(1)+\mu(2)\mu(5)\mu(10)&\\
=1+(-1)+(-1)+1=0&.\\
\end{align*}

Полное значение функции Мебиуса должно стать очевидным со следующей теоремой.\\

\begin{theorem} 
	\label{Теорема 6--7.}
	 (Формула Инверсии Мебиуса). {\it Пусть $F$ и $f$ теоретико-числовые~~ функции, связанные формулой}\\

\begin{align*}
F(n)&=\sum_{d|n}f(d).\\
\intertext{Затем}
f(n)&=\sum_{d|n}\mu(d)F(n/d)=\sum_{d|n}\mu(n/d)F(d).
\end{align*}
\end{theorem}
\begin{proof} Две суммы, упомянутые в заключении теоремы считаются одинаковыми при замене фиктивного индекса $d$ на $d'=n|d$; как $d$ включает все положительные делители $n$, так и $d'$.\\

Проведя необходимые вычисления, получим\\

\[
(1)~~~\sum_{d|n}\mu(d)F(n/d)=\sum_{d|n}\left(\mu(d)\sum_{c|(n/d)}f(c)\right)=\sum_{d|n}\left(\sum_{c|(n/d)}\mu(d)f(c)\right).
\]
Легко проверить, что  $d|n$ и $c|(n/d)$ тогда и только тогда $c|n$ и $d|(n/c)$.Из-за этого последнее выражение в (1) становится\\
\begin{align*}
(2)~~~~~~~~\sum_{d|n}\left(\sum_{c|(n/d)}\mu(d)f(c)\right)&=
\sum_{c|n}\left(\sum_{d|(n/d)}f(c)\mu(d)\right)\\
&=\sum_{c|n}\left(f(c)\sum_{d|(n/c)}\mu(d)\right).\\
\end{align*}
В соответствии с Теоремой 6--6, сумма $\sum_{d|(n/c)}\mu(d)$ должна быть равна нулю, за исключением $n/c=1$ (то есть, кода $n=c$), в этом случае он равен 1; в результате правая часть (2) упрощается до \\
\[
\sum_{c|n}\left(f(c)\sum_{d|(n/c)}\mu(d)\right)=\sum_{c=n}f(c)\cdot 1=f(n),
\]
давая нам заявленный результат.
\end{proof}

Давайте снова используем $n=10$, чтобы проилюстрировать, как двойная сумма в (2) оборачивается. В данном случае мы находим, что

\begin{align*}
\sum_{d|10}\left(\sum_{c|(10/d)}\mu(d)f(c)\right)&=
\mu(1)[f(1)+f(2)+f(5)+(10)]\\
&+\mu(2)[f(1)+f(5)]+\mu(5)[f(1)+f(2)]+\mu(10)f(1)\\
&=f(1)[\mu(1)+\mu(2)+\mu(5)+\mu(10)]\\
&+f(2)[\mu(1)+\mu(5)]+f(5)[\mu(1)+\mu(2)]+f(10)\mu(1)\\
&=\sum_{c|10}\left(\sum_{d|(10/c)}f(c)\mu(d)\right).\\
\end{align*}

Чтобы увидеть как раблтает инверсия в конкретном случае,мы напомним читателю, что функции $\tau$ и $\sigma$ обе могут быть описаны как ``сумма функций'':\\
\[\tau(n) = \sum \limits_{d\mid n} 1
\text{\quad and\quad}
\sigma(n) =\sum \limits_{d\mid n} d\]
Теорема 6--7 показывает нам что эти формулы могут быть инвертированны чтобы получить\\
\[
1=\sum_{d|n}\mu(n/d)\tau(d)
\text{\quad and\quad} 
n=\sum_{d|n}\mu(n/d)\sigma(d),\\
\]
допустимо для всех $n\ge1$.\\

Теорема 6--4 гарантирует, что если $f$ мультипликативная функция,то, так же как и
$F(n)=\sum_{d|n}f(d)$.
Перевернув ситуацию, можно было спросить является ли мультипликативная природа $F$ силами, которые из $f$. 
Довольно удивительно, это именно то, что происходит.\\[5mm]
\begin{theorem}
\label{ Теорема 6--8.} {\it Если $F$ мультипликативная функция и}\\
\[
F(n)=\sum_{d|n}f(d),\]\\
{\it тогда $f$ также мультипликативна.}\end{theorem}
\begin{proof}
Пусть $m$ и $mn$ относитьно простые числа. Вспомним что любой делитель $d$ из $mn$ может быть однозначно записан как $d=d_1d_2$, где $d_1|m,~d_2|n,$ и НОД$(d_1{,} d_2)=1$. Таким образом, используя формулу инверсии,\\
\begin{align*}
f(mn)&=\sum_{d|mn}\mu(d)F\left(\frac{mn}d \right)\\
&=\sum_{\begin{matrix} d_1 \\ d_2 \end{matrix}|\begin{matrix} m \\ n \end{matrix}}\mu(d_1d_2)F\left(\frac{mn}d \right)\\
&=\sum_{\begin{matrix} d_1 \\ d_2 \end{matrix}|\begin{matrix} m \\ n \end{matrix}}
\mu(d_1)\mu(d_2)F\left(\frac m{d_1}\right)F\left(\frac n{d_2}\right)\\
&=\sum_{d_1|m}\mu(d_1)F\left(\frac m{d_1}\right)\sum_{d_2|n}\mu(d_2)F\left( \frac n{d_2}\right)=f(m)f(n),\\
\end{align*}
что и является утверждением теоремы. Излишне говорить, что мультипликативные  $\mu$ и $F$ имеют решающее значение для приведённого выше расчёта.
\end{proof}

	\begin{center}
	\LARGE {\textsf {\textbf {ПРОБЛЕМЫ 6.2}}}\\[5mm]
\end{center}
\begin{enumerate}
\item 	\begin{enumerate}
\item Для каждого положительного целого числа $n$, покажите, что\\
\[\mu(n)\mu(n+1)\mu(n+2)\mu(n+3)=0.\]
\item Для любого числа $n\ge3$,показать что
$\sum_k^n=_1\mu(k!)=1.$
\end{enumerate}
\item {\it Функция Мангольда} $\Lambda$ определяется следующим образом\\
\[
\Lambda(n)=\begin{cases}
\log{p},&\text{если $n=p^k$, где $p$ простое $k\ge1$} \\
0,&\text{в противном случае}
\end{cases}\]\\
Доказать, что $\Lambda(n)=\sum_{d|n}\mu(n/d)\log{d}=-\sum_{d|n}\mu(d)\log{d}.$ [{\it Подсказка:} ~Покажите, что $\sum_{d|n}\Lambda(d)=\log{n}$ а затем, примените формулу инверсии Мёбиуса.]
\item Пусть $n=p_1^{k_1}p_2^{k_2\dots}p_r^{k_r}$ факторизация простого числа $n>1$.\\
 Если $f$ мультипликативная функция, докажите, что\\
\[
\sum_{d|n}\mu(d)f(d)=(1-f(p_1))(1-f(p_2))^{\dots} (1-f(p_r)).
\]
[{\itshape Подсказка:}\: По Теореме 6-4, функция $F$ определяется $F(n)=\sum_{d|n}\mu(d)f(d)$\\
 является мультипликативной; следовательно, $f(n)$ является произведением $F(p_i^{k_1})$.]
\item Если целое число $n>1$ имеет простую факторизацию $n=p_1^{k_1}p_2^{k_2\dots}p_r^{k_r}$, используете Проблему 3 чтобы установить следующее:
\begin{enumerate}
	\item $\sum_{d|n}\mu(d)\tau(d)=(-1)^r;$
	\setlength{\parskip}{3mm}
	\item $\sum_{d|n}\mu(d)\sigma(d)=(-1)^rp_1p_{2~} ^{~\dots}p_r;$
	\item $\sum_{d|n}\mu(d)\d=(1-1/p_1)(1-1/p_2)^{\dots}(1-1/p_r);$
	\item $\sum_{d|n}d\mu(d)=(1-p_1)(1-p_2)^{\dots}(1-p_r).$
\end{enumerate}
\item Пусть $S(n)$обозначает число бесквадратных делителей $n$.~ Установим, что\\
\[
S(n)=\sum_{d|n}|\mu(d)|=2^r
\]
где $r$ число простых делителей $n$. [{\itshape Подсказка:} $S$ это мультипоикативная функция.] 
\item Найдите формулы для 
$\sum_{d|n}\mu^2(d)/\tau(d)$ и
$\sum_{d|n}\mu^2(d)/\sigma(d)$ в терминах простой факторизации  $n$.
\item The {\it Функция Лиовиля $\lambda$} определяется $\lambda(1)=1$ и $\lambda(n)=(-1)^{k_1+k_2+\dots+k_r},$\\
если простая факторизация $n>1$ равна $n=p_1^{k_1}p_2^{k_2\dots}p_r^{k_r}.$ Например,\\
$\lambda(360)=\lambda(2^3 \cdot 3^2 \cdot5)=(-1)^{3+2+1}=(-1)^6=1.$
\begin{enumerate}
\item Докажите, что $\lambda$ является мультипликативной функцией.
\item Учитывая положительное число $n$,убедитесь, что\\
\[\sum_{d|n}\lambda(d)=\begin{cases}
1,&\text{Если $n=m^2$ для некоторого целого числа $m$}\\
0,&\text{в противном случае}
\end{cases} \]
\end{enumerate}
\item Если целое число $n>1$ имеет простую факторизацию $n=p_1^{k_1}p_2^{k_2\dots}p_r^{k_r}$, установить\\
что $\sum_{d|n}\mu(d)\lambda(d)=2^r.$
	
	
	
	
	
\end{enumerate}	
\end{document}