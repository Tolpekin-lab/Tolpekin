\documentclass[11pt]{article}
\usepackage{amsmath,amssymb,amsthm}
\usepackage{algorithm}
\usepackage[noend]{algpseudocode} 

%---enable russian----

\usepackage[utf8]{inputenc}
\usepackage[russian]{babel}

% PROBABILITY SYMBOLS
\newcommand*\PROB\Pr 
\DeclareMathOperator*{\EXPECT}{\mathbb{E}}


% Sets, Rngs, ets 
\newcommand{\N}{{{\mathbb N}}}
\newcommand{\Z}{{{\mathbb Z}}}
\newcommand{\R}{{{\mathbb R}}}
\newcommand{\Zp}{\ints_p} % Integers modulo p
\newcommand{\Zq}{\ints_q} % Integers modulo q
\newcommand{\Zn}{\ints_N} % Integers modulo N

% Landau 
\newcommand{\bigO}{\mathcal{O}}
\newcommand*{\OLandau}{\bigO}
\newcommand*{\WLandau}{\Omega}
\newcommand*{\xOLandau}{\widetilde{\OLandau}}
\newcommand*{\xWLandau}{\widetilde{\WLandau}}
\newcommand*{\TLandau}{\Theta}
\newcommand*{\xTLandau}{\widetilde{\TLandau}}
\newcommand{\smallo}{o} %technically, an omicron
\newcommand{\softO}{\widetilde{\bigO}}
\newcommand{\wLandau}{\omega}
\newcommand{\negl}{\mathrm{negl}} 

% Misc
\newcommand{\eps}{\varepsilon}
\newcommand{\inprod}[1]{\left\langle #1 \right\rangle}

 
\newcommand{\handout}[5]{
  \noindent
  \begin{center}
  \framebox{
    \vbox{
      \hbox to 5.78in { {\bf Научно-исследовательская практика} \hfill #2 }
      \vspace{4mm}
      \hbox to 5.78in { {\Large \hfill #5  \hfill} }
      \vspace{2mm}
      \hbox to 5.78in { {\em #3 \hfill #4} }
    }
  }
  \end{center}
  \vspace*{4mm}
}

\newcommand{\lecture}[4]{\handout{#1}{#2}{#3}{Scribe: #4}{Number-Theoretic Functions #1}}

\newtheorem{theorem}{Теорема}
\newtheorem{lemma}{Лемма}
\newtheorem{definition}{Определение}
\newtheorem{corollary}{Следствие}
\newtheorem{fact}{Факт}

% 1-inch margins
\topmargin 0pt
\advance \topmargin by -\headheight
\advance \topmargin by -\headsep
\textheight 8.9in
\oddsidemargin 0pt
\evensidemargin \oddsidemargin
\marginparwidth 0.5in
\textwidth 6.5in

\parindent 0.4in
\parskip 0ex

\begin{document}

\lecture{}{Лето 2020}{}{Толпекин Максим}
 
Since $\mu$ is known to be a multiplicative function, an appeal to
Theorem 6-4 is legitimate; this result guarantees that $F$ is multiplicative
too.~~Thus, if the canonical factorization of $n$ is $n=p^{k_1}_1 p^{k_2\dots}_2 p^{k_r}_r$,
then $F(n)$ is the product of the values assigned to $F$ for the prime powers in this representation:
$$F(n)=F(p^{k_1}_1)F(p^{k_2}_2)\dots F(p^{k_r}_r)=0.$$
\begin{flushleft} We record this result as\\[5mm] \end{flushleft}
{\setlength{\leftskip}{9mm}
	\setlength{\rightskip}{9mm}
\noindent{\sc Theorem 6--6.}
{\it For each positive integer $n\geqslant 1$,}
$$\sum_{d|n}\mu(d)=\begin{cases}
1&\text{if $n=1$} \\
0&\text{if $n>1$}
\end{cases}
$$\\
{\it where d runs through the positive divisors of n.}\\

}

{For an illustration of this last theorem consider $n=10$. The divisors of 10 are 1, 2, 5, 10 and the desired sum is}\\
\begin{align*}
\sum_{d|10}\mu(d)=\mu(1)+\mu(2)\mu(5)\mu(10)&\\
=1+(-1)+(-1)+1=0&.\\
\end{align*}

The full significance of M\"obius' function shold become apparent with the next theorem.\\

{\setlength{\leftskip}{9mm}
	\setlength{\rightskip}{9mm}
\noindent{\sc ~~Theorem 6--7} (M\"obius Inversion Formula). {\it Let $F$ and $f$ be two number-theoretic~~ functions related by the formula}\\

}
\begin{align*}
F(n)&=\sum_{d|n}f(d).\\
\intertext{Then}
f(n)&=\sum_{d|n}\mu(d)F(n/d)=\sum_{d|n}\mu(n/d)F(d).
\end{align*}
{\it Proof:~~} The two sums mentioned in the conclusions of the theorem are seen to be the same upon replacing the dummy index $d$ by $d'=n|d$; as $d$ ranges over all positive divisors of $n$, so does $d'$.\\

Carrying out the required computation, we get\\

$$
(1)~~~\sum_{d|n}\mu(d)F(n/d)=\sum_{d|n}\left(\mu(d)\sum_{c|(n/d)}f(c)\right)=\sum_{d|n}\left(\sum_{c|(n/d)}\mu(d)f(c)\right).
$$
It is easily verified that $d|n$ and $c|(n/d)$ if and only if $c|n$ and $d|(n/c)$. Because of this, the last expression in (1) becomes\\
\begin{align*}
(2)~~~~~~~~\sum_{d|n}\left(\sum_{c|(n/d)}\mu(d)f(c)\right)&=
\sum_{c|n}\left(\sum_{d|(n/d)}f(c)\mu(d)\right)\\
&=\sum_{c|n}\left(f(c)\sum_{d|(n/c)}\mu(d)\right).\\
\end{align*}
In compliance with Theorem 6--6, the sum $\sum_{d|(n/c)}\mu(d)$ must vanish except when $n/c=1$ (that is, when $n=c$), in which case it is equal to 1; the upshot is that the right-hand side of (2) simplifies to\\
\begin{align*}
\sum_{c|n}\left(f(c)\sum_{d|(n/c)}\mu(d)\right)=\sum_{c=n}f(c)\cdot 1=f(n),
\end{align*}
giving us the stated result.\\

Let us use $n=10$ again to illustrate how the double sum in (2) is turned around. In this instance, we find that

\begin{align*}
\sum_{d|10}\left(\sum_{c|(10/d)}\mu(d)f(c)\right)&=
\mu(1)[f(1)+f(2)+f(5)+(10)]\\
&+\mu(2)[f(1)+f(5)]+\mu(5)[f(1)+f(2)]+\mu(10)f(1)\\
&=f(1)[\mu(1)+\mu(2)+\mu(5)+\mu(10)]\\
&+f(2)[\mu(1)+\mu(5)]+f(5)[\mu(1)+\mu(2)]+f(10)\mu(1)\\
&=\sum_{c|10}\left(\sum_{d|(10/c)}f(c)\mu(d)\right).\\
\end{align*}

To see how M\"obius inversion works in a particular case, we remind the reader that the functions $\tau$ and $\sigma$ may both be described as ``sum function'':\\
$$\tau(n) = \sum \limits_{d\mid n} 1
\text{\quad and\quad}
\sigma(n) =\sum \limits_{d\mid n} d$$
Theorem 6--7 tells us that these formulas may be inverted to give\\
$$
1=\sum_{d|n}\mu(n/d)\tau(d)
\text{\quad and\quad} 
n=\sum_{d|n}\mu(n/d)\sigma(d),\\
$$
valid for all $n\ge1$.\\

Theorem 6--4 insures that if $f$ is a multiplicative function, then so is
$F(n)=\sum_{d|n}f(d)$.
Turning the situation around, one might ask whether the multiplicative nature of $F$ forces that of $f$. 
Suprisingly enough, this is exactly what happens.\\[5mm]
{\sc Theorem 6--8.} {\it If $F$ is a multiplicative function and}\\
$$F(n)=\sum_{d|n}f(d),$$\\
{\it then $f$ is also multiplicative.}\\[5mm]
{\itshape Proof:}\: Let $m$ and $mn$ can be uniquely written as $d=d_1d_2$, where $d_1|m,~d_2|n,$ and gcd$(d_1{,} d_2)=1$. Thus, using the inversion formula,\\
\begin{align*}
f(mn)&=\sum_{d|mn}\mu(d)F\left(\frac{mn}d \right)\\
&=\sum_{\begin{matrix} d_1 \\ d_2 \end{matrix}|\begin{matrix} m \\ n \end{matrix}}\mu(d_1d_2)F\left(\frac{mn}d \right)\\
&=\sum_{\begin{matrix} d_1 \\ d_2 \end{matrix}|\begin{matrix} m \\ n \end{matrix}}
\mu(d_1)\mu(d_2)F\left(\frac m{d_1}\right)F\left(\frac n{d_2}\right)\\
&=\sum_{d_1|m}\mu(d_1)F\left(\frac m{d_1}\right)\sum_{d_2|n}\mu(d_2)F\left( \frac n{d_2}\right)=f(m)f(n),\\
\end{align*}
which is the assertion of the theorem. Needless to say, the multiplicative character of $\mu$ and of $F$ is crucial to the above calculation.

	\begin{center}
	\LARGE {\textsf {\textbf {PROBLEMS 6.2}}}\\[5mm]
\end{center}
\begin{enumerate}
\item 	\begin{enumerate}
\item For each positive integer $n$, show that\\
$$\mu(n)\mu(n+1)\mu(n+2)\mu(n+3)=0.$$
\item For any integer $n\ge3$, show that
$\sum_k^n=_1\mu(k!)=1.$
\end{enumerate}
\item The {\it Mangoldt function} $\Lambda$ is defined by\\
$$\Lambda(n)=\begin{cases}
\log{p},&\text{if $n=p^k$, where $p$ is aprime and $k\ge1$} \\
0,&\text{otherwise}
\end{cases}$$\\
Prove that $\Lambda(n)=\sum_{d|n}\mu(n/d)\log{d}=-\sum_{d|n}\mu(d)\log{d}.$ [{\it Hint:} ~First show that $\sum_{d|n}\Lambda(d)=\log{n}$ and then apply the M\"obius Inversion Formula.]
\item Let $n=p_1^{k_1}p_2^{k_2\dots}p_r^{k_r}$ be the prime factorization of the integer $n>1$.\\
 If $f$ is a multiplicative function, prove that\\
$$\sum_{d|n}\mu(d)f(d)=(1-f(p_1))(1-f(p_2))^{\dots} (1-f(p_r)).$$
[{\itshape Hint:}\: By the Theorem 6-4, the function $F$ defined by $F(n)=\sum_{d|n}\mu(d)f(d)$\\
 is multiplicative; hence, $f(n)$ is the product of the values $F(p_i^{k_1})$.]
\item If the integer $n>1$ has the prime factorization $n=p_1^{k_1}p_2^{k_2\dots}p_r^{k_r}$, use Problem 3 to establish the following:
\begin{enumerate}
	\item $\sum_{d|n}\mu(d)\tau(d)=(-1)^r;$
	\setlength{\parskip}{3mm}
	\item $\sum_{d|n}\mu(d)\sigma(d)=(-1)^rp_1p_{2~} ^{~\dots}p_r;$
	\item $\sum_{d|n}\mu(d)\d=(1-1/p_1)(1-1/p_2)^{\dots}(1-1/p_r);$
	\item $\sum_{d|n}d\mu(d)=(1-p_1)(1-p_2)^{\dots}(1-p_r).$
\end{enumerate}
\item Let $S(n)$ denote the number of sqeare-free divisors of $n$.~ Establish that\\
$$S(n)=\sum_{d|n}|\mu(d)|=2^r$$
where $r$ is the number of distinct prime divisors of $n$. [{\itshape Hint:} $S$ is a multiplicative function.] 
\item Find formulas for 
$\sum_{d|n}\mu^2(d)/\tau(d)$ and
$\sum_{d|n}\mu^2(d)/\sigma(d)$ in terms of the prime\\ factorization of $n$.
\item The {\it Liouville $\lambda$-function} is defined by $\lambda(1)=1$ and $\lambda(n)=(-1)^{k_1+k_2+\dots+k_r},$\\
if the prime factorization of $n>1$ is $n=p_1^{k_1}p_2^{k_2\dots}p_r^{k_r}.$ For instance,\\
$\lambda(360)=\lambda(2^3 \cdot 3^2 \cdot5)=(-1)^{3+2+1}=(-1)^6=1.$
\begin{enumerate}
\item Prove that $\lambda$ is a multiplicative function.
\item Given a positive integer $n$, verify that\\
$$\sum_{d|n}\lambda(d)=\begin{cases}
1,&\text{if $n=m^2$ for some integer $m$}\\
0,&\text{otherwise}
\end{cases} $$
\end{enumerate}
\item If the integer $n>1$ has the prime factorization $n=p_1^{k_1}p_2^{k_2\dots}p_r^{k_r}$, establish\\
that $\sum_{d|n}\mu(d)\lambda(d)=2^r.$
	
	
	
	
	
	
	
	
	
	
	
	
\end{enumerate}	
\end{document}